\documentclass[times, utf8, diplomski]{fer}
\usepackage{booktabs}
\usepackage{etoolbox}
\usepackage{listings}
\usepackage{hyperref}
\usepackage{multicol}
\usepackage[table]{xcolor}
\usepackage{adjustbox}
\usepackage{wrapfig,lipsum,booktabs}

% Define labeled quote.
\newcounter{numquote}
\newenvironment{lquote}[1]{%
  \stepcounter{numquote}%
  \expandafter\xdef\csname#1\endcsname{\fbox{\thenumquote}}%
  \quote``\ignorespaces}{\unskip''\fbox{\thenumquote}\endquote}
\newcommand\quoteref[1]{\csname#1\endcsname}

\begin{document}

% TODO: Navedite broj rada.
\thesisnumber{2927}

% TODO: Navedite naslov rada.
\title{Izrada upravljačkih programa za ugradbene uređaje u tehnologiji modernog C++-a}

% TODO: Navedite vaše ime i prezime.
\author{Hrvoje Bolešić}

\maketitle

% Ispis stranice s napomenom o umetanju izvornika rada. Uklonite naredbu \izvornik ako želite izbaciti tu stranicu.
% \izvornik

% Dodavanje zahvale ili prazne stranice. Ako ne želite dodati zahvalu, naredbu ostavite radi prazne stranice.
\zahvala{}

\tableofcontents

\chapter{Uvod}

Sve je bitnije proizvesti bolji i kvalitetniji softver. Svaki korisnik želi što bolju moguću aplikaciju, iako je definicja \textit{"što je bolje"} drugačija za svakog pojedinca, nerijetko uključuje efikastnost kao jedan od glavnih parametara. Uz sve napore struke činjenica je ta da je softver sve manje efikasan iako, prema \textit{Moore}-ovom zakonu\engl{Moore's Law}\cite{mooresLaw}, bi trebalo biti baš obrnuto. \textit{Moore}-ov zakon govori o tome kako se broj tranzistora u integriranim krugovima\engl{Integrated Circuits, IC} udvostručava svake dvije godine. Iz toga bi bilo logično zaključiti kako sami integrirani krugovi postaju efikasniji i time da i programi izvršavani na njima bivaju također efikasniji. Na žalost, nije tako jednostavno, u stvarnosti ispada da iako hardver postaje efikasniji, softver je sve manje efikasan. O tome govori, tzv., \textit{Wirth}-ov zakon, koji govori kako brzina software pada više no što brzina hardvera raste. \textit{Wirth}-ov zakon:
\begin{quote}
    ``Software is getting slower more rapidly than hardware is getting faster.''\cite{wirthsLaw}
\end{quote}
Dodatno, ne govori samo o brzini softvera već i o njegovoj kompleksnosti koja višestruko raste u usporedbi s kompleksnosti hardvera. Objašnjenje zašto se ovo događa leži u činjenici kako današnje aplikacije nude funkcionalnosti koje nisu striktno vezane za glavnu funkcionalnost prvotne aplikacije. Tako na primjer, danas će mnogi uređivači teksta zahtjevati internetsku vezu iako sama primarna funkcionalnost uređivanja teksta nema potrebu za internetskom vezom. Ovo je tipičan primjer gdje aplikacija nudi dodatne funkcionalnosti koje imaju dodatne ovisnosti, konkretno u primjeru internetsku vezu, i time uvode dodatnu kompleksnost u izradu samog proizvoda na benefit korisnika koji dobiva više mogućnosti. Samim uvođenjem dodatnih mogućnosti krajnji proizvod postaje memorijski, a vrlo često i procesivno zahtjevniji. \\ 

Dodatna i možda bitnija činjenica zašto softver postaje sve kompleksniji je obilje, često specijaliziranog, hardvera dostupnog za korištenje unutar aplikacija. Poznato je kako smo pred krajem \textit{Moore}-ova zakona, odnosno činjenicom da se tranzistori više neće moći fizički smanjiti kako bi se mogao udvostručiti broj tranzistora na jednom integriranom krugu. Buduća ubrzanja su jedino moguća tako da se posao raspodijeli na više procesora, odnosno, došli smo do samih limita što je moguće ostvariti na jednom procesoru. Samim time logično je očekivati samo još više specijaliziranih IC-ova unutar budućih platformi kako bi se koristili za izradu aplikacija. Direktna je implikacija da će softver time biti još kompleksniji, što je i za očekivati po \textit{Wirth}-ovom zakonu. \\ 

Pitanje je \textit{"Kako se boriti sa kompleksnošću?"}, \textit{"Da li je moguće potpuno elimirati kompleksnost?"}. Na žalost, iskustva pokazuju da je kompleksnost nužna, odnosno moderan softver zahtjeva kompleksna rješenja. Međutim, struka je razvila alate koji mogu pomoći oko kompleksnosti. Do sada vjerojatno najbolji alat za suzbijanje komplesknosti mora biti \textbf{abstrakcija}\engl{Abstraction}. Iako se pojam abstrakcije gotovo uvijek spominje u kontesktu Objektno orjentiranog programiranja\engl{Object Oriented Programing, OOP}, sami koncept nije nužno vezan samo za OOP, međutim nedvojbeno je da koncept ima svoje začetke u Objektno orjentiranoj paradigmi\engl{Object oriented paradigm}. Abstrakcija je proces izgradnje jednostavnijeg modela kompleksnih jedinica tako da sadrži dovoljno informacije za ispunjenje zadatka u određenom kontekstu. \\   

Od dva navedena razloga koja objašnjavaju zašto \textit{Writh}-ov zakon vrijedi, prvi se za konektst ovog diplomskog rada može zanemariti i protumačiti kao tzv. \textit{Code Bloat}, odnosno kompleksnost koja je uvedena zbog dodavanja "nepotrebnog" koda. Drugi razlog je mnogo interesantniji i intrigantniji jer se radi o problemu tehničke naravi, ključnim za ovaj diplomksi rad. Kako je predhodno rečeno razlog povečanja kompleksnosti softvera je dodavanje specijaliziranog hardvera za kojeg je najčešće potrebno izgraditi upravljački program za jednostavniju uporabu uređaja. Konkretno ovaj diplomski rad promatra kako se nositi sa kompleksnosti i izazovima izrade upravljačkih programa za moderne aplikacije na ugradbenim uređajima. Činjenica o tome da softver generalno postaje kompleksniji je možda najviše vidljiva upravo u domeni ugradbenih uređaja, pogotvo tamo gdje se pokušava izgraditi generično riješenje. Moderan softver u domeni ugradbenih uređaja nužno zahtjeva modernije tehnologije nego što je trenutna praksa kako bi se sama domena programske potpore za ugradbene uređaje nastavila razvijati.
\pagebreak

\section{Što je ugradbeni uređaj?}
Definicija ugradbenog uređaja se neprestano mijenja kako se razvijaju novi proizvodi s novim izazovima. Prema \textit{Jack G. Ganssle}-ovom i \textit{Michael Barr}-ovom riječniku, \textit{Embedded Systems Dictionary}\cite{ganssle2003embedded}, definicija ugradbenog uređaja je:
\begin{quote}
    ``A combination of computer hardware and software, and perhaps additional mechanical or other parts, designed to perform a dedicated function. In some cases, embedded systems are part of a larger system or product, as in the case of an antilock braking system in a car.''
\end{quote}
% Kombinacija računalong hardvera i softvera i po potrebi dodatnih mehaničkih i/ili drugih dijelova, osmišljenih da izvršavaju odrđenu funkciju. U nekim slučajevima ugradbeni sustavi su dio većeg sistema ili proizvoda kao u slučaju automobilskog ABS sustava. 
Definicija je osmišljena 2003. godine i za to vrijeme vrlo dobro definira ugradbene uređaje, međutim, danas ne obuhvaća mnoge proizvode koji se smatraju ugradbenim uređajima. Ponajviše ova definicija ne obuhvaća uređaje koje nemaju samo jednu dediciranu funkciju, danas ugradbeni uređaji nerijetko imaju i više funkcija. Neke ličnosti u domeni ugradbenih uređaja također navode ovu definiciju međutim pokušavaju je upotpuniti ili preoblikovati. Jedan od utjecajnijih ličnosti u domeni ugradbenih uređaja, \textit{Dan Saks}, daje svoju jednostavniju definiciju\cite{danSaksWritingBetterEmbeddedSoftware}:
\begin{quote}
    ``The job of a computer in an embedded system is to be something other than a general-purpose computer.''
\end{quote}
Ono što pokušava izreći ovom definicijom je da je ugradbeni uređaj bilo koji uređaj za koje se nebi smatralo da je računalo. Iako manje precizna definicija, bolje služi u razumjevanju o čemu je riječ kada se govori o ugradbenim uređajima. Drugi ljudi, poput tvorca jezika C++, \textit{Bjarne Straustrup}-a, se ne upuštaju u striktno definiranje\cite{bjarneC++ForEmbeddedSystems} ugradbenih uređaja već se više oslanjaju na sličnosti u razvoju softvera među raznim uređajima i priča o razvoju softvera za ugradbene uređaje na osnovi te sličnosti. Kako bi dodatno istaknuli da je definiranje ugradbenih uređaja težak zadatak, jedan od autora prije navedenog riječnika, \textit{Jack. G. Ganssle}, se 2018.\cite{embeddedDotComWhatsEmbedded} godine osvrnuo na danu definiciju i iskazao svoj skepticizam oko nje te tražio čitatelje za sugestije. \\
Za potrebe ovog diplomskog rada odabiremo definiciju bliže \textit{Dan Saks}-u, te definiramo ugradbeni uređaj kao bilo koji uređaj koji nije smatran računalom generične namjene.     
\pagebreak

\section{Što je upravljački program?}
Upravljački programi\engl{Device Drivers} su\cite{linuxDeviceDrivers}:
\begin{quote}
    ``They(device drivers) are distinct "black boxes" that make a particular piece of hardware respond to a well-defined internal programming interface; they hide completely the details of how the device works.''
\end{quote}
Gornja definicija definira značenje upravljačkog programa u kontekstu \textit{Linux} operacijskog sustava. Općenito operacijski sustavi zahtjevaju upravljačke programe za pojedini hardver kako bi ga mogli utilizirati u samom radu sustava. Ono što je možda prešutno u ovoj defniciji je to da sam operacijski sustav definira jedno sučelje za široki spektar različitih uređaja, što je nužno kako bi se olakšala interakcija između operacijskog sustava i uređaja. Ukoliko bi svaki uređaj imao zasebno sučelje operacijski sustavi bi bili vrlo glomazni i bili bi još kompliciraniji za sami razvoj. Međutim, činjenica da svi uređaji implementiraju jedno definirano sučelje je u ovoj definiciji izrečena samo zato što se govori o upravljačkim programima u kontekstu operacijskih sustava. Za potrebe ugradbenih uređaja nije nužno da upravljački programi imaju identično sučelje, štoviše, u većini slučajeva dizaj kakav se koristi za implementaciju upravljačkih programa u domeni operacijskih sustava bi donio nepotrebnu kompleksnost, nedeterminističnost, vremensku i prostornu složenost, sve odlike nepoželjne u domeni ugradbenih uređaja. \\
Svi navedeni nedostatci su često razlog zašto se u razvoju programske podrške za ugradbene uređaje ne koristi tehnologija operacijskih sustava te se sam program izvodi na tzv. "golom metalu"\engl{Bare Metal}, odnsono program se izvodi na procesoru bez dodatnih posrednika poput operacijskih sustava, virtualnih mašina, interpretera i sl. Upravljački programi se svakako mogu gledati kao abstrakcija hardvera i utoliko, kao i što sama definicija govori, moraju biti sposobni donekle sakriti korišteni uređaj. Ovdje se ponovno vidi korištenje abstrakcije kao alata za suzbijanje kompleksnosti, ukoliko nebi postojao koncept upravljačkih programa, neposredno korištenje uređaja bi unosilo dodatnu kompleksnost u izradu softvera za ugradbeni uređaj. U kontekstu ovog diplomskog rada upravljački programi poštuju gore navedenu definiciju osim što nisu namjenjeni da imaju jedinstveno sučelje jer bi ta činjenica dovela do pesimizacije krajnjeg softvera. Ova pesimizacija je sasvim razumljiva i poželjna u području generičnog računarstva, međutim vrlo je kobna i često se izbjegava u domeni ugradbenih uređaja zbog gore navedenih razloga. 
\pagebreak

\section{Moderni C++?}
\begin{quote}
    ``C++ is a language for defining and using light-weight abstractions. It has significant strengths in areas where hardware must be handled effectively and there are significant complexity to cope with. This includes many resource constrained systems and much foundational and infrastructure code.''
\end{quote}
Gornja definicija\cite{bjarneStroustrupC++} je data od samog tvorca programskog jezika C++, \textit{Bjarne Stroustrup}-a, i odgovara na pitanje "\textit{što je C++?}". Jezik je napravljen kao nasljednik C programskog jezika s kojim je i danas kompatibilan, gotovo svi C programi su validni C++ programi. C++ posuđuje koncepte iz začetnika objektno orjentirane paradigme, programski jezik Simula, te odlučuje koncepte poput razreda, sučelja, nasljeđivanja i sl. ponuditi korisnicima kako bi kreirali abstrakcije na korisničkoj razini koji su u mnogočemu ekvipotentni abstrakcijama ugrađenim direktno u sam jezik. Ovo je vrlo močan koncept gdje se korisniku omogućuje da efektivno može poširiti jezik sa vlastitim abstrakcijama bez ulaženja u internu implementaciju i standardizaciju jezika. Doduše nije C++ jedini jezik koji omogućava kreiranje abstrakcija na korisničkoj razini, međutim interna filozofija jezika C++ ga odvaja od drugih jezika. Glavna filozofija jezika C++, koja na neki način vodi daljnji razvoj jezika, može se ukratko izreči kao\cite{theDesignAndEvolutionOfC++}:
\begin{quote}
    ``What you don't use, you don't pay for.''
\end{quote}
Gornji citat je ono što razlikuje C++ od toliko drugih jezika, ono što se pokušava izreči se još zove i \textbf{Zero-Overhead} princip, točnije govori se o tome kako je jezik koncipiran na način da korisnik može napisati program tako da "ne plača" ništa drugo osim onoga što mu je potrebno za implementaciju softvera. Ovo se čini možda suvišno za izreči ali u praksi se mnogo puta ispostavlja da su određena svojstva jezika povezana i ukoliko korisnik želi koristiti jedno svojstvo nužno povlači i drugo o kojem ovisi čak iako u trenutnom kontekstu nema logičke ovisnosti, već samo dizajnerske ovisnosti. Ono što se pokušava reći da se dizajneri mnogih jezika odlučuju dizajnirati jezična svojstva na način da određena svojstva jezika nisu međusobno isključiva čak iako logički to jesu. Štoviše, takav dizajn često dovodi do vremenske i prostorne složenosti na korist smanjenja korisničke kompleksnosti. C++ kao jezik koji se reklamira kao moguća opcija u razvoju operacijskih sustava, domeni ugradbenih uređaja, autoindustriji, itd., nije dizajniran tako da "žrtvuje" vremensku i prostornu složenost za smanjenje kompleksnosti. To naravno ne znači kako jezik ne nudi nikakve mehanizme za suzbijanje kompleksnosti, nego da pesimiziranje vremenske i prostorne složenosti nije prihvatljivo kako bi se smanjila kompleksnost. \\

Dakako, ništa od predhodno navedenog ne odgovara na pitanje "\textit{što je moderni C++?}". Moderni C++ je često vidljiva sintagma u C++ zajednici i šire koja implicira da postoji modernija inačica jezika C++, to je istina, međutim to ne znači da je to potpuno drugi jezik. Sam jezik C++ nastao je 1985. godine, a 1998. je prvi put standardiziran kao \textit{ISO}\footnote{International Organization for Standardization} standard, iz toga je vidljivo da sam jezik ne asocira na riječ "moderan" jer potječe iz 80-tih godina prošlog stojeća. Međutim, recentniji standardi jezika(2011. godina nadalje) su revitalizirali jezik sa novim, sigurnijim i sve u svemu boljim svojstvima koje se uvelike oslanjaju na moderne tehnike statičke analize i povijesno dobre prakse. Ugrubo gledajući, sintagma "moderan C++" govori o jeziku C++ nakon standarda uvedenog 2011. godine. Dakako standardizacija jezika C++ nije stala i još uvijek je procesu izgradnje međutim C++11 standard se generalno uzima kao početak "modernog C++"-a. Osim C++11 standarda, postoje C++14, C++17 i najnoviji C++20 standard. Za potrebe ovog rada koriste se jezična svojstva iz svih navedenih standarda međutim ne spominje se eksplicitno za svako svojstvo iz kojeg standarda potječe jer je ta činjenica suvišna za kontekst ovog rada.


\chapter{Moderne tehnologije u programskoj potpori za ugradbene računalne sustave}
Stanje danas u razvoju programske potpore za ugradbene sustave je takvo da je još uvijek daleko najpopularnija tehnologija programski jezik C. O ovoj činjnici govori \textit{Dan Saks} u svojoj konferenciskoj prezentaciji\cite{danSaksWritingBetterEmbeddedSoftware}, gdje jasno prikazuje trendove najpopularnijih tehnologija u domeni ugradbenih uređaja gdje C neupitno prednjači nad bilo kojom drugom tehnologijom. Gledajući retrospektivno kroz povijest može se naslutiti zašto je i danas, u 2022. godini, tehnologija izbora mnogih inženjera u domeni ugradbenih uređaja još uvijek programski jezik C, jednostavno povijesno gledano C je bio dugo vremena prisutan i ustalio se u inženjerskoj praksi kao jedini pravi odgovor. Toliko se ustalio u inženjerskom poslu da ga inženjeri diljem svijeta, u različitim domenama, nazivaju \textit{Lingua Franca}. Dodatno, iz tehničke perspektive, programski jezik C je savršeno dizajniran tako da inženjer ima potpunu kontrolu nad svojim programom, te ostavlja vrlo malo prostora za neku drugu tehnologiju, osim možda asemblerskog koda platforme, uz to postoji i C prevodilac za gotovo svaku postojeću platformu. Postoji ogromna količina softvera napisanih u programskom jeziku C, a neki su i napisani kako bi pospiješili produktivnost samo jezika. Gotovo svi popularni operacijski sustavi danas svoje jezgre imaju razvijene uz pomoć C-a, što direktno povlači i činjenicu da postoji potražnja za C programerima. Programski jezik C je fenomen u računarskoj povijesti i vjerojatno će još dugo vremena biti "zajednički nazivnik" svih tehnologija u računarstvu. Možda najveći razlog zašto C neće tako skoro biti zamjenjen nekom drugom tehnologijom je dizajnerska odluka jezika da jezik vjerno prikazuje platformu na kojoj se izvršava. Ukoliko ništa od navedenog ne uvjerava čitatelja kako jezik C tu da ostane, onda je tu činjenica da nove generacije inženjera ne mogu ništa izgubiti s poznavanjem programskog jezika C, što znači da je moguće da će u budućnosti biti korišten u obrazovne svrhe. \\
Sa svim tim rečenim, programski jezik C ima mane i ovaj diplomski rad smatra kako postoje bolje, moderne, alterntive u pojedinim kontekstima. Ne opovrgava kako je programski jezik C validan izbor u mnogim slučajevima, međutim tvrdi da mnogi programeri koji danas koriste C, za svoj trenutni projekt bi trebali koristiti nešto drugo, nešto modernije. Također, prije prelaska u raspravu o modernim tehnologijama, vrijedi napomentuti kako programski jezik C nije jedina starija tehnologija korištena u domeni ugradbenih uređaja, ali je toliko dominantna da se može ugrubo reći da je jedina. Za kontekst ovog rada možemo reći kako je programski jezik C predstavnik starih tehnologija za razvoj programske podrške u domeni ugradbenih uređaja. Moderne tehnologije su zasnovane na prijašnjima kao što je to uvijek istina u inženjerskoj praksi - iteriranjem nad starim idejama dolazimo do novih. Kao i za stare tehnologije, kako bi se olakšala rasprava, za svrhu ovog diplomskog rada odabran je predstavnik modernih tehnologija, očekivano, programski jezik C++, odnosno moderna inačica jezika C++. Ovaj odabir nije slučajan, već je izabran iz razloga što je druga najpopularnija tehnologija navedena u konferenciskoj prezentacija \textit{Dan Saks}-a\cite{danSaksWritingBetterEmbeddedSoftware} upravo programski jezik C++. Iako možda izgleda šaljivo, kroz ovaj diplomski rad pokušati će se prezenirati izazovi i rješenja modernih tehnologija u izradi upravljakih programa u ugradbenim uređajima tako što će se uspoređivati odabrani predstavnici starih i novih tehnologija, programski jezik C i C++. Ovo je napravljeno kako bi se pokušalo doći do konkretnih problema i rješenja koja konkretna moderna tehnologija, ovdje C++, rješava. Osim programskog jezika C++ kao predstavnika modernih tehnologija, vrijedi spomenuti vrlo obečavajuću modernu tehnologiju, programski jezik Rust, koja se predstavlja kao tehnologija koja će zamjeniti programski jezik C. 

\section{Potrebe modernih sustava}

\section{Usporedba tehnologija}

\chapter{Izrada upravljačkog programa}
Srž ovog diplomskog rada je izraditi i analizirati upravljački program za određeni periferni uređaj ugradbenog uređaja. Izrađeni upravljački program će biti uspoređen sa implementacijom napisanom u programskom jeziku C od strane proizvođača samog uređaja, ta činjnica bi trebala osigurati najbolju kvalitetu upravljačkog programa. Ekspresivnost izvornog koda je bitan faktor u izradi programskog proizvoda i bitan u ovom diplomskom radu jer je pretpostavka kako su moderne tehnologije ekspresivnije jer imaju bolju sposobnost izricanja inženjerske namjere. Međutim, vrlo je teško objektivno kvatificirati ekspresivnost izvornog koda, tako da će faktor ekspresivnosti biti bez objektivne kvatifikacije, ostavljen na čitatelju da sam procijeni ekspresivnost modernih naspram starih tehnologija. Osim ekspresivnosti izvornog koda, objektivna svojstva od interesa za ovaj diplomski rad su definitivno vrijeme izvođenja i veličina krajnjeg programa. Pretpostavka je kako se ova dva svojstva ne razlikuju između odabranih tehnologija i time dokazuje kako mehanizmi modernih tehnologija ne uvode pesimizaciju ni u jednom od odabranih svojstva. 

\section{Opis platforme}
Platforma koja se koristi za provedbu ovog diplomskog rada je razvojna pločica proizvedena od strane kompanije \textit{Aconno} koja koristi mikrokontroler kompanije \textit{Nordic Semiconductors}. Točan mikrokontroler je \textit{nRF52840} koji je vrlo popular u \textit{IoT}\engl{Internet of things} domeni, prvenstveno zbog podrške bežične komunikacije putem popularnih protokola poput \textit{Wi-Fi}, \textit{Bluetooth}, \textit{BLE}\engl{Bluetooth Low Energy}, itd. Osim mogućnosti bežične komunikacije mikrokontroler je obskrbljen sa mnogo dodatne periferije za procesiranje signala, kriptografiju, serijsku komunikaciju, sinkronizaciju, itd. Sam mikroprocesor je izgrađen po \textit{ARM Cortex-M4} arhitekturi koja dolazi s ugrađenom \textit{FPU}\engl{Floating Point Unit} jedinicom i sklopovljem za otklanjanjem grešaka, te podržava \textit{ARMv7-M} instrukcijski set. \\
Gore navedeni uređaj je odabran zbog svoje široko podržane \textit{ARM} arithekture, vrlo dobre podrške putem \textit{Nordic Semiconductors} foruma i zajednice inženjera, mnogobrojne i močne periferije, lakoće samog programiranja, podrške modernog prevodioca specifično izgrađenim za \textit{ARM} procesore i mogućnosti korištenja sklopovlja za otklanjanje pogrešaka. Sa razvoj programa za sami ugradbeni uređaj koristi se osobno računalo sa operacijskim sustavom \textit{Linux} i ostalim alatima. 

\section{Alati}
Za samo učitavanje krajnjeg programa na ugradbeni uređaj koristi se program komandne linije \textit{nrfjprog} pruženog od strane proizvođača samog mikrokontrolera. Alat je vrlo jednostavan za korištenje i za svoj rad zahtjeva samo spajanje razvojne pločice sa osobnim računalom putem USB kabla. Za razvoj samog izvornog koda koristi se razvojno okruženje \textit{CLion} od tvrtke \textit{JetBrains} koji je vjerojatno jedino dobro razvojno okruženje dostupno na \textit{Linux} operacijskom sustavu za razvoj programa uz pomoć programskog jezika C/C++. Verzioniranje koda je ostvareno uz pomoć alata \textit{git} koji je danas, \textit{de facto}, standard za verzioniranje. Iako \textit{git} dolazi kao komandno linijski alat većinom je korištena ekstenzija razvojnog okruženja \textit{CLion} koja indirektno koristi terminalni alat. Verzioniranje koda je bitna sastavnica projekta, međutim ovaj diplomski rad nema veliku povijest iz čega se da zaključiti kako je alat \textit{git} korišten vrlo oskudno. Veći faktor od do sada spomenutih tehnologija je defnitivno alat za izgradnju programa, \textit{CMake}. \textit{CMake} je ekstenzivno korišten, pokušavajući poštovati moderne smijernice koje dolaze sa modernijim verzijama alata. Valja ispraviti grešku i reči kako nije točno reći da je \textit{CMake} alat za izgradnju alata, već je \textit{CMake} generator za alat za izgradnju alata, što komplicira mentalnu sliku o tome što \textit{CMake} jest, međutim to je istina. \textit{CMake} je korišten kako bi se nadomijestilo stari, još i danas popularan alat, \textit{Make}, koji je poput programskog jezika C, dobar alat, međutim današnji alati nude bolje mehanizme za izgradnju projekata. \textit{CMake} u ovom diplomskom radu interno koristi \textit{Make} kao krajnji alat za izgradnju programa, međutim sam projekt nema nikakve veze sa alatom \textit{Make} te se potencijalno može koristi drugi alati za izgranju programa poput, danas popularnog, \textit{Ninja}-e.\\
Svi dosada nevedeni alati su striktno pomoćni alati u razvoju softvera, alat bez kojega bi svi ovi alati bili suvišni jest prevodilac\engl{Compiler}. Točnije sam prevodilac je alat koji se sastoji od niza alata, zato se ponekad naziva i lanac alata\engl{Toolchain}. U okviru ovog diplomskog rada koristi se \textit{GNU}, C i C++ prevodilac, točnije \textit{ARM GNU} lanac alata verzije 10.3\footnote{Link na stranicu prevodioca: \hyperref[https://developer.arm.com/downloads/-/gnu-rm]{https://developer.arm.com/downloads/-/gnu-rm}}. Ovo je vrlo moderna verzija \textit{GNU} prevodioca koja je potrebna za kontekst ovog diplomskog rada kako bi se pokazale mogućnosti modernog C++. Kako bi se izrađeni upravljački mogao uspoređivati s nečim, dodatno je bilo potrebno uvesti ovisnost o \textit{nRF5 SDK}-u\engl{Software Development Kit} koji sadrži implementacije svog softvera za \textit{Nordic Semiconductors} proizvode, pa tako i za \textit{nRF52840} mikrokontroler. Uz \textit{Nordic Semiconductors} SDK kako bi realizirali povratnu informaciju, sa ugradbenog uređaja nazad na osobno računalo, dodatno je uvedena ovisnost o \textit{Segger RTT} modulu koji omogućava korištenje standardnih funkcija poput \textbf{\textit{printf}} tako da se poruke proslijede osobnom računalu koje koristi alat \textit{JLinkRTTViewer} kao serijski monitor za prihvačanje proslijeđenih poruka. Na taj način korisnik može lakše analizirati ponašanje samog ugradbenog uređaja. Ukoliko je potrebno, koristi se \textit{GDB}\engl{GNU Debugger} zajedno sa \textit{JLinkGDBServer}-om kako bi se zaista duboko analizrala sama egzekucija samog mašinskog koda. Kako bi se analizirali krajnji produkti, objektni kod i krajnji program, koristili su se alati iz paketa alata \textit{binutils} koji dolazi zajedno s \textit{GNU} lancom alata. 

\chapter{Rezultati}
Something.

\chapter{Zaključak}
Ukoliko pogledamo \hyperref[table:results]{tablicu} rezultata vidljivo je kako je veličina programa u određenom modu prevođenja izgrađena uz pomoć programskog jezika C++ uvijek proporcionalno manja s obzirom na tehnologiju programskog jezika C. Ovi rezultati ne govore o tome da je to istina za svaki slučaj međutim pobija teze koje govore da su programi izgrađeni uz pomoć jezika C++ uvijek veći od funkcijski ekvivalentnih C programa. Ovdje se vidi snaga ekspresije modernog programskog jezika koja omogućava provodiocu da elminira određene dijelove koda te omogućava ekstenzivnije i agresivnije optimizacije. S druge strane ako se pogledaju prosječna vremena izvođenja vidljivo je kako programski jezik C još uvijek prednjači programskom jeziku C++ u \textit{Debug} modu prevođenja. Ovo je za očekivati jer programski jezik C++ je nešto kompleksniji od programskog jezika C i time gubi na vremenu izvođenja bez uključivanja optimizatora na višim razinama optimizacije. Kada se osbosobi optimizator u \textit{Release} modu prevođenja može se vidjeti kako su prosječna vremena izvođenja identična za slučajeve kada se upravljački program koristi u modu s iznimkama. Upravljački program u modu s radnim čekanjem uvijek biva najsporija opcija što se vrlo vjerojatno može prepisati samom dizajnu takvog moda, a ne samoj tehnologiji izvedbe. Radno čekanje se opčenito izbjegava jer troši procesorske resurse, a kao što je ovdje i pokazano, ne daje zadovoljavajuće rezultate. Vjerojatno su se zato i dizajneri unutar \textit{Nordic Semiconductors} kompanije i odlučili kako neće podržati mod radnog čekanja unutar upravljačkog programa za generator slučajnih brojeva, međutim tom odlukom je implicitno rečeno da mikrokontroler mora koristiti mehanizam iznimki koji ponekad nije dostupan ili nedopušten. Na stranu sa implementacijom upravljačkog programa u modu s radnim čekanjem, oba upravljačka programa u modu s inznikama daju jednako vrijeme izvođenja, odnosno mala greška se ovdje pripisuje nedeterminističnosti uređaja za generiranje slučajnih brojeva i zaključujemo kako su vremena izvođenja identična. Ovime se pokazuje kako se korištenjem pomno odabranih modernih tehnologija ne žrtvuje vrijeme izvođenja i kako korisnik može imati potpunu moć da time što će na kraju prevodilac emitirati u krajnji izvršni program. Ono što još valja reči kako je kvaliteta i ekspresivnost koda puno bolja u modernih tehnologija, ovo svojstvo nije kvantificirano, međutim čitatelja se snažno potiče da pregleda izvorne kodove programa i procijeni sam. Ono što valja reći je to da je istina kako moderne tehnologije, naručito programski jezik C++, jesu kompliciranije u odnosu na tehnologije poput programskog jezika C, međutim upravo zbog te kompleksnosti daju mogućnost ekspresije koju čak i prevodilac može razumjeti i na osnovu tih informacija generirati bolji, sigurniji i manji program. Svojstvo ekspresivnosti kroz koncept abstrakcije je svojstvo koje moderne tehnologije nude u odnosu na stare i to će se u budućnosti samo perpeturati kako bi se stare tehnologije istisnule iz domena koje bi stvarno trebale koristiti nešto modernije. Na kraju treba reći kako će programski jezik C uvijek imati mjesto u budučnosti računarstva međutim, kako to obično biva, inžinjeri moraju osvjestiti kako programski jezik C nije uvijek najbolji alat za sve probleme, pogotovo u domeni ugradbenih uređaja, i kako postoje bolji, moderniji alati, poput programskog jezika C++. 

% Not cited literature.
\nocite{danSaksTalkingToCProgrammersAboutC++}
\nocite{cppreference}
\nocite{latexguide}
\nocite{dansakshowigothere}
\nocite{nrfsdk}
\nocite{compilerexplorer}
\nocite{cmake}

\bibliography{literatura}
\bibliographystyle{fer}

\begin{sazetak}
Rad se bavi usporedbom starih i mordenih tehnologija u domeni ugradbenih uređaja. Kao predstavnika starih tehnologija uzima se programski jezik C, dok se za predstavnika modernih tehnologija uzima moderna inačica programskog jezika C++. Na primjeru izrađivanja upravljačkog programa za ugradbeni uređaj se analiziraju pojedina svojstva predstavljenih tehnologija te se ističe kako je za današnje potrebe softvera u domeni ugradbenih uređaja potrebno koristi ekpresivnije i fleksibilnije tehnologije. Kao objektivni pokazatelji s kojim se odabrane tehnologije mogu uspoređivati odabiru se vrijeme izvođenja te veličina samog programa. Dodatno se raspravlja o kvaliteti, čitljivosti i ekspresivnosti samog izvornog koda.

\kljucnerijeci{C++, C, Ugradbeni uređaj, Programska potpora, Upravljački program, Moderne thenologije}
\end{sazetak}

% TODO: Navedite naslov na engleskom jeziku.
\engtitle{Construction of Drivers for Embedded Devices in Modern C++ Technology}
\begin{abstract}
This thesis is trying to compare old and modern technologies in the domain of embedded systems. As a representatitve of old technologies it takes programming language C, while picking a modern variant of programming language C++ as a representative of modern technologies. By implementing a driver for an embedded system of choice, this thesis analyses individual features of the representative technologies and it asserts the need for more expressive and flexible technologies for modern solutions inside of the domain of embbeded systems. As objectively measured attributes, which the representatives are compared by, this thesis chose execution time and program size. Additionaly it talks about source code quality, readability and the expressivness of the choosen technologies.  

\keywords{C++, C, Embedded device, Software, Device driver, Modern technology}
\end{abstract}

\end{document}
