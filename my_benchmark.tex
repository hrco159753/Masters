\chapter{Izrada upravljačkog programa}
Srž ovog diplomskog rada je izraditi i analizirati upravljački program za određeni periferni uređaj ugradbenog uređaja. Izrađeni upravljački program će biti uspoređen sa implementacijom napisanom u programskom jeziku C od strane proizvođača samog uređaja, ta činjnica bi trebala osigurati najbolju kvalitetu upravljačkog programa. Ekspresivnost izvornog koda je bitan faktor u izradi programskog proizvoda i bitan u ovom diplomskom radu jer je pretpostavka kako su moderne tehnologije ekspresivnije jer imaju bolju sposobnost izricanja inženjerske namjere. Međutim, vrlo je teško objektivno kvantificirati ekspresivnost izvornog koda, tako da će faktor ekspresivnosti biti bez objektivne kvantifikacije, ostavljen na čitatelju da sam procijeni ekspresivnost modernih naspram starih tehnologija. Osim ekspresivnosti izvornog koda, objektivna svojstva od interesa za ovaj diplomski rad su definitivno vrijeme izvođenja i veličina krajnjeg programa. Pretpostavka je kako se ova dva svojstva ne razlikuju između odabranih tehnologija i time dokazuje kako mehanizmi modernih tehnologija ne uvode pesimizaciju ni u jednom od odabranih svojstva. 

\section{Opis platforme}
Platforma koja se koristi za provedbu ovog diplomskog rada je razvojna pločica proizvedena od strane kompanije \textit{Aconno} koja koristi mikrokontroler kompanije \textit{Nordic Semiconductors}. Točan mikrokontroler je \textit{nRF52840} koji je vrlo popular u \textit{IoT}\engl{Internet of things} domeni, prvenstveno zbog podrške bežične komunikacije putem popularnih protokola poput \textit{Wi-Fi}, \textit{Bluetooth}, \textit{BLE}\engl{Bluetooth Low Energy}, itd. Osim mogućnosti bežične komunikacije mikrokontroler je obskrbljen sa mnogo dodatne periferije za procesiranje signala, kriptografiju, serijsku komunikaciju, sinkronizaciju, itd. Sam mikroprocesor je izgrađen po \textit{ARM Cortex-M4} arhitekturi koja dolazi s ugrađenom \textit{FPU}\engl{Floating Point Unit} jedinicom i sklopovljem za otklanjanjem grešaka, te podržava \textit{ARMv7-M} instrukcijski set. \\
Gore navedeni uređaj je odabran zbog svoje široko podržane \textit{ARM} arithekture, vrlo dobre podrške putem \textit{Nordic Semiconductors} foruma i zajednice inženjera, mnogobrojne i močne periferije, lakoće samog programiranja, podrške modernog prevodioca specifično izgrađenim za \textit{ARM} procesore i mogućnosti korištenja sklopovlja za otklanjanje pogrešaka. Za razvoj programa za sami ugradbeni uređaj koristi se osobno računalo sa operacijskim sustavom \textit{Linux} i ostalim alatima. 

\section{Alati}
Za samo učitavanje krajnjeg programa na ugradbeni uređaj koristi se program komandne linije \textit{nrfjprog} pruženog od strane proizvođača samog mikrokontrolera. Alat je vrlo jednostavan za korištenje i za svoj rad zahtjeva samo spajanje razvojne pločice sa osobnim računalom putem USB kabla. Za razvoj samog izvornog koda koristi se razvojno okruženje \textit{CLion} od tvrtke \textit{JetBrains} koji je vjerojatno jedino dobro razvojno okruženje dostupno na \textit{Linux} operacijskom sustavu za razvoj programa uz pomoć programskog jezika C/C++. Verzioniranje koda je ostvareno uz pomoć alata \textit{git} koji je danas, \textit{de facto}, standard za verzioniranje. Iako \textit{git} dolazi kao komandno linijski alat većinom je korištena ekstenzija razvojnog okruženja \textit{CLion} koja indirektno koristi terminalni alat. Verzioniranje koda je bitna sastavnica projekta, međutim ovaj diplomski rad nema veliku povijest iz čega se da zaključiti kako je alat \textit{git} korišten vrlo oskudno. Veći faktor od do sada spomenutih tehnologija je defnitivno alat za izgradnju programa, \textit{CMake}. \textit{CMake} je ekstenzivno korišten, pokušavajući poštovati moderne smjernice koje dolaze sa modernijim verzijama alata. Valja ispraviti grešku i reći kako nije točno reći da je \textit{CMake} alat za izgradnju alata, već je \textit{CMake} generator za alat za izgradnju programa, što komplicira mentalnu sliku o tome što \textit{CMake} jest, međutim to je istina. \textit{CMake} je korišten kako bi se nadomijestilo stari, još i danas popularan alat, \textit{Make}, koji je poput programskog jezika C, dobar alat, međutim današnji alati nude bolje mehanizme za izgradnju projekata. \textit{CMake} u ovom diplomskom radu interno koristi \textit{Make} kao krajnji alat za izgradnju programa, međutim sam projekt nema nikakve veze sa alatom \textit{Make} te se potencijalno može koristi drugi alati za izgradnju programa poput, danas popularnog, \textit{Ninja}-e.\\
Svi dosada nevedeni alati su striktno pomoćni alati u razvoju softvera, alat bez kojega bi svi ovi alati bili suvišni jest prevodilac\engl{Compiler}. Točnije sam prevodilac je alat koji se sastoji od niza alata, zato se ponekad naziva i lanac alata\engl{Toolchain}. U okviru ovog diplomskog rada koristi se \textit{GNU}, C i C++ prevodilac, točnije \textit{ARM GNU} lanac alata verzije 10.3\footnote{Link na stranicu prevodioca: \hyperref[https://developer.arm.com/downloads/-/gnu-rm]{https://developer.arm.com/downloads/-/gnu-rm}}. Ovo je vrlo moderna verzija \textit{GNU} prevodioca koja je potrebna za kontekst ovog diplomskog rada kako bi se pokazale mogućnosti modernog C++. Kako bi se izrađeni upravljački program mogao uspoređivati s nečim, dodatno je bilo potrebno uvesti ovisnost o \textit{nRF5 SDK}-u\engl{Software Development Kit} koji sadrži implementacije svog softvera za \textit{Nordic Semiconductors} proizvode, pa tako i za \textit{nRF52840} mikrokontroler. Uz \textit{Nordic Semiconductors} SDK kako bi realizirali povratnu informaciju, sa ugradbenog uređaja nazad na osobno računalo, dodatno je uvedena ovisnost o \textit{Segger RTT} modulu koji omogućava korištenje standardnih funkcija poput \textbf{\textit{printf}} tako da se poruke proslijede osobnom računalu koje koristi alat \textit{JLinkRTTViewer} kao serijski monitor za prihvaćanje proslijeđenih poruka. Na taj način korisnik može lakše analizirati ponašanje samog ugradbenog uređaja. Ukoliko je potrebno, koristi se \textit{GDB}\engl{GNU Debugger} zajedno sa \textit{JLinkGDBServer}-om kako bi se zaista duboko analizralo samo izvršavanje mašinskog koda. Kako bi se analizirali krajnji produkti, objektni kod i krajnji program, koristili su se alati iz paketa alata \textit{binutils} koji dolazi zajedno s \textit{GNU} lancom alata. 

\section{Postav}\label{section:setup}
Projekt ovog dimplomskog rada je osmišljen kao jednostavna okolina za mjerenje\engl{Benchmark envirnoment}. Baza projekta je zapravo program koji predstavlja jedno okruženje za mjerenje, a kontekst koji se mjeri se uvodi kao ovisnost putem primitiva funkcije zajedno sa dodanim svojstvima samog mjerenja. Točnije baza projekta je izvršni program koji ovisi o funkciji čije će vrijeme izvršavanja biti mjereno i potom ispisano prosječno vrijeme izvršavanja, broj ciklusa i totalno vrijeme izvođenja. Kako bi pomnije objasnili kako izgleda sama baza projekta pogledajmo izvorni kod datoteke u kojoj se nalazi ulazna točka programa.
\lstset{language=C++, tabsize=2, frame=single, breaklines=true, showstringspaces=false}
\begin{lstlisting}
#include <cstdio>
    // `std::printf`
#include <Benchmark.hpp>
    // `BenchmarkDescriptor`, `benchmark`
#include <Config.hpp> 
    // `name`, `repeat`, `run`
#include <Stopwatch.hpp>
    // `Stopwatch`

static const Benchmark::BenchmarkDescriptor 
    descriptor{name(), repeat(), &run};

int main() {
    Benchmark::Meta::Stopwatch auto 
        stopwatch{Stopwatch::Stopwatch{}};

    if (not stopwatch.init()) 
    {
        std::printf("Can't init stopwatch!");
        return 1;
    }

    const auto [totalTime]
        {Benchmark::benchmark(stopwatch, descriptor)};
    const auto [name, cycles, dummy]
        {descriptor};

    std::printf(
        "Name: %10s\n\tTotal time: %luus\n\tCycles: %lu\n\tAverage: %luus\n",
        name.data(), 
        static_cast<std::uint32_t>(totalTime.count()),
        static_cast<std::uint32_t>(cycles),
        static_cast<std::uint32_t>(totalTime.count() / cycles)
    );
}
\end{lstlisting}
U datoteci je vidljvo nekoliko interesantnih dijelova, jezgra cijelog programa je enkapsulirana unutar razreda \textbf{\textit{Benchmark::BenchmarkDescriptor}} koji opisuje kontekst koji će biti mjeren. Konstrukcija ovog tipa zahtijeva ime mjerenog konteksta, broj ponavljanja izvršavanja te samu funkciju koja će se izvršavati. Sve informacije za inicijalizaciju instance \textbf{\textit{Benchmark::BenchmarkDescriptor}} su uvedene putem funckija \textbf{\textit{name}}, \textbf{\textit{repeat}} i \textbf{\textit{run}}. Same deklaracije funkcija su navedene u zaglavlju \textbf{\textit{Config.hpp}}, međutim same implementacije su odvojene od baznog programa kako bi mogli izgraditi više programa sa drugim kontekstima za mjerenje. U kasnijem poglavlju su izdvojene konkretne implementacije za navedene funkcije ovisno o kojem kontekstu je riječ.\\
Slijedeća bitna sastavnica baznog izvršnog programa je štoperica\engl{Stopwatch} koja je enkapsulirana unutar razreda \textbf{\textit{Stopwatch}} te je implementirana kao adapter na već postojeći upravljački program za timer koji je implementiran od strane \textit{Nordic Semiconductors} kompanije. Razred \textbf{\textit{Stopwatch}} se koristi samo u kontekstu ovog diplomskog rada i zato sama implementacija nije toliko ekstenzivna i prenosiva jer nije bilo potrebe za ponovnu uporabu. Prvotna ideja diplomskog rada nije uopće uključivala praćenje vremena na samom ugradbenom uređaju već je bila ideja eksternalizirati praćenje na osobnom računalu ili sl. Međutim, kroz vrijeme razvoja se ispostavilo kako su brzine izvođenja programa bile prevelike i kako bi greška sa eksternaliziranim praćenjem vremena bila prevelika, štoviše, toliko velika da se ne bi mogao izvući nikakav objektivan zaključak iz dobivenih mjerenja. Zbog toga je odlučeno praćenje vremena internalizirati na ugradbenom uređaju uz pomoć perifernog uređaja timera. Dodatno, kako bi se olakšalo praćenje vremena implementacija razreda \textbf{\textit{Stopwatch}} interno koristi već postojeći upravljački program implementiran od strane \textit{Nordic Semiconductors} kompanije. Altrenativa je bila samostalno izgraditi novi upravljački program, međutim time bi se još dodatno zakomplicirao razvoj cijelog projekta pa je odlučeno koristiti već izgrađeno sučelje za periferni uređaj timer-a.\\
Komponenta glavnog programa koja zapravo odrađuje izvršavanje funckcije konteksta, \textbf{\textit{run}}, jest poziv funkcije \textbf{\textit{Benchmark::benchmark}} čija je deklaracija definirana u zaglavlju \textbf{\textit{Benchmark.hpp}}. Funkcija nakon mjerenja vraća totalno vrijeme potrebno za izvođenje te se potom to vrijeme u glavnoj funkciji dijeli sa brojem ciklusa kako bi se dobilo prosječno vrijeme izvođenja i poptom se ispisuje pomoću funkcije \textbf{\textit{std::printf}}. Bitno je ovdje napomenuti kako je u okruženju ovog programa nadomješten sistemski poziv funkcije \textbf{\textit{write}} s implementacijom od strane \textit{Segger RTT} sistemskog poziva. Zbog ove promjene funkcija \textbf{\textit{std::printf}} proslijeđuje poruku, putem \textit{RTT} protokola, prema osobnom računalu kako bi korisnik lakše mogao pregledati rezultate mjerenja. Za detalje, čitatelja se potiče da pogleda točne postavke projekta kako bi saznao točno kako je ovo napravljeno jer takvi detalji nisu bitni za kontekst rada. \\
Valja podiskutirati i analizirati samu implementaciju funkcije \textbf{\textit{Benchmark::benchmark}} jer se ispostavlja kako nije trivijalna za implementirati. Izvorni kod funkcije je dan u nastavku.
\lstset{language=C++, tabsize=2, frame=single, breaklines=true, showstringspaces=false}
\begin{lstlisting}
Benchmark::BenchmarkResult Benchmark::benchmark(
    PolymorphicStopwatch stopwatch, 
    const BenchmarkDescriptor &benchmarkDescriptor
) noexcept 
{
    Benchmark::BenchmarkResult result{.totalTime = std::chrono::microseconds{}};

    warmup(1'000);

    for (std::size_t i{0}; i < benchmarkDescriptor.repeat; ++i) {
        stopwatch.begin();
        benchmarkDescriptor.run();
        result.totalTime += stopwatch.time();
    }

    return result;
}
\end{lstlisting}
U implementaciji funkcije je vidljiva varijabla \textbf{\textit{result}} koja je vraćena kao konačna vrijednost iz funkcije, a prvotno je inicijalizrana sa vrijednostću \textbf{\textit{0ms}} te se unutar petlje taj broj povećava za vrijeme izvođenja jednog izvršavanja konteksta, odnosno funckije \textbf{\textit{run}} koja je doustupna putem \textbf{\textit{benchmarkDescriptor}}-a. Ono što je zanimljivo u ovoj implementaciji je poziv funkcije \textbf{\textit{warmup(1'000)}}. Ova funkcija je napravljena kako bi se mjerenje moglo odraditi još preciznije bez mogućnosti utjecaja procesorskih komponenti poput prediktora granjanja. U prošlim iteracijama funckije \textbf{\textit{Benchmark::benchmark}} nije postojao ovaj poziv i time je dolazilo do nedeterminističkih mjerenja koja su bila uočljiva na testom primjeru. Kako bi objasnili kako konstruirati primjer, odnosno konteskt mjerenja pogledajmo testni primjer.\\
Testni primjer se sastoji od dvije datoteke izvornog koda, jedan koji je napisan koristeći programski jezik C, dok je drugi napisan koristeći programski jezik C++. Doljnji lijevi primjer prikazuje testni primjer napisan u programskom jeziku C, dok desni primjer prikazuje ekvivalentni primjer napisan u tehnologiji C++.
\pagebreak
\begin{multicols}{2}
\lstset{language=C, tabsize=2, frame=single, breaklines=true, showstringspaces=false, basicstyle=\small}
\begin{lstlisting}    
void run()
{
    for (size_t i = 0U; i < 100; ++i)
    {
        __asm__ volatile(
            "nop"
        );
    }
}

const char *name() 
{ 
    return "C::Test"; 
}
size_t repeat() 
{ 
    return 10000; 
}      
\end{lstlisting}
\columnbreak
\lstset{language=C++, tabsize=2, frame=single, breaklines=true, showstringspaces=false, basicstyle=\small}
\begin{lstlisting}    
void run()
{
    for (size_t i = 0U; i < 100; ++i)
    {
        asm volatile(
            "nop"
        );
    }
}

const char *name() 
{
    return "C++::Test"; 
}
size_t repeat() 
{ 
    return 10000; 
}      
\end{lstlisting}
\end{multicols}
Testni primjer je konstruiran na način da bude identičan u obije tehnologije kako bi dokazali da je izkonstruirani sustav za mjerenje validan. Dodatno je pregledan izgenerirani asemblerski kod koji je identičan u oba slučaja tako da se sa sigurnošću može reći kako su primjeri identični, odnosno \textbf{\textit{run}} funckije. Uz identične \textbf{\textit{run}} funkcije gotovo su identične i ostale dvije funkcije međutim postoji mala razlika u imenu kako bi se raspoznao jedan kontekst od drugoga. Neki čitatlji će se pitati zašto se u jednoj verziji koristi \textbf{\textit{\_\_asm\_\_}}, a u drugoj \textbf{\textit{asm}}, te zašto se uoće koristi takav konstrukt. Odgovor na pitanje zašto se koristi je taj da ukoliko bi ostavili \textbf{\textit{for}} petlju praznom prevodilac bi potpuno izostavio(optimizirao) petlju i funkcija efektivno ništa ne bi radila. Kako bi to izbjegli koristi se tzv. \textit{inline assembly} kako bi se spriječilo prevodioca da optimizira petlju. Nadalje, odgovor na pitanje zašto se razlikuju sintakse je jednostavno taj da programski jezik C nema ključnu riječ \textbf{\textit{asm}} već \textbf{\textit{\_\_asm\_\_}}, međutim funkcionalno za ovaj primjer rade identičnu stvar.\\
Nakon što je objašnjeno zašto i kako je izkonstruiran testni primjer valja reći kako su vremenski rezultati za oba programa, neovisno o tehnologiji, bili identični. Dakle, sustav za mjerenje je validan i može se koristiti za daljnja mjerenja. Ukoliko bi se vremena izvođenja drastično razlikovala između dvaju verzija to bi značilo kako sustav za mjerenje nije napravljen dobro, srećom to ovdje nije slučaj. 

\section{Upravljački program}
Upravljački program koji je izrađen u sklopu ovog diplomskog rada je upravljački program za generator slučajnih brojeva\engl{Random number generator}. Uređaj za generiranje slučajnih brojeva je nadasve jednostavan uređaj sa svega nekoliko funkcija međutim dovoljno je kompleksan za kontekst ovog diplomskog rada. Sam uređaj je mapiran u memoriju mikrokontrolera te sadrži nekoliko bitnih registara, registar za pokretanje generatora slučajnih brojeva, registar za zaustavljanje generatora slučajnih brojeva, registar za pohranu slučajnog broja, registar za zastavicu koja govori da li je slučajan broj spreman za čitanje, odnosno da li je slučajan broj izgeneriran do kraja i na kraju registar za omogućavanje iznimke kada je slučajan broj izgeneriran. Iz poslijednja dva navedena registra može se zaključiti kako sam uređaj može raditi na dva načina, uz pomoć mehanizma iznimaka i uz pomoć radnog čekanja. U nastavku pogledajmo relevantno sučelje upravljačkog programa koje nudi \textit{nRF5 SDK}. 

\lstset{language=C, tabsize=2, frame=single, breaklines=true, showstringspaces=false}
\begin{lstlisting}
typedef void (* nrfx_rng_evt_handler_t)(uint8_t rng_data);
typedef struct
{
    bool     error_correction : 1;
    uint8_t  interrupt_priority;
} nrfx_rng_config_t;

nrfx_err_t nrfx_rng_init(nrfx_rng_config_t const * p_config, nrfx_rng_evt_handler_t handler);
void nrfx_rng_start(void);
void nrfx_rng_stop(void);
void nrfx_rng_uninit(void);

void nrf_rng_int_enable(uint32_t mask);
void nrf_rng_int_disable(uint32_t mask);
bool nrf_rng_int_get(nrf_rng_int_mask_t mask);
uint8_t nrf_rng_random_value_get(void);
          
\end{lstlisting}
Dva definirana tipa \textbf{\textit{nrfx\_rng\_evt\_handler\_t}} i \textbf{\textit{nrfx\_rng\_config\_t}} su tipovi koji definiraju prototip funkcije koja može primiti novo izgenerirani slučajni broj i tip koji enkapsulira postavke upravljačkog programa pri inicijalizaciji. Slijedeće četiri funkcije dosta očito služe za inticijalizaciju, pokretanje, zaustavljanje i deinitcijalizaciju uređaja. Isto tako slijedeće četiri funkcije u nastavku pružaju funkcionalnost dozvoljavanja/nedozvoljavanja iznimke kada se broji izgenerira, dohvaćanje zastavice koja govori o tome da li je iznimka omogućena ili onemogućena te poslijednje za dohvaćanje same slučajno generirane vrijednosti. Sučelje kakvo je prikazano ostvaruje svoju zadaću, pruža korisniku funkcionalnost generiranja slučajnih brojeva putem iznimaka, međutim ne postoji direktan način za ostvariti dohvaćanje slučajnog broja koristeći radno čekanje. Ovime se korisnika prisiljava na korištenje mehanizma iznimki čak iako to nije njegova želja, time ovaj upravljački program biva manje fleksibilan. Dodatno bitno je uočiti kako navedene strukture i funkcije ne mogu biti enkapsulirane u razred ili slično jer odabrana tehnologija za implementaciju, programski jezik C, ne podržava takve mehanizme. \\
S druge strane pogledajmo relevantno sučelje upravljačkog programa implementiranog u programskom jeziku C++.
\lstset{language=C++, tabsize=2, frame=single, breaklines=true, showstringspaces=false}
\begin{lstlisting}

namespace Rng
{

enum class Policy { Polling, Interrupt };

template <Policy kPolicy = Policy::Interrupt>
class Driver {
    /*--- Help type ---*/
    struct ConditionStruct {
        bool isDone() const noexcept;
        void waitOn() const noexcept;
    };

    /*--- Constructor ---*/
    Driver() noexcept;

    /*--- Methods ---*/
    std::uint8_t 
    get() noexcept 
    requires(kPolicy == Policy::Polling);
    
    template <std::size_t N>
    std::array<std::uint8_t, N> 
    get() noexcept 
    requires(kPolicy::value == Policy::Polling);

    ConditionStruct 
    set(
        std::invocable<std::uint8_t> auto &userHandler
    ) noexcept
    requires(kPolicy::value == Policy::Interrupt);
};
}
              
\end{lstlisting}
Na prvi pogled izgleda intimidirajuće sučelje, međutim puno je ekspresivnije i sigurnije za korištenje u odnosu na prije pokazani upravljački uređaj. Prvo valja napomenuti kako je u ovom slučaju upravljački uređaj realiziran uz pomoć obrasca razreda koji kao svoj jedini argument prima mod rada upravljačkog programa\engl{Policy} i tu odluku ostavlja na samom korisniku i time postaje više ekstenzivan od svojeg konkurenta. Odlukom korisnika o modu rada, korisnik raspolaže sa pripadajućim sučeljem, zapravo se u ovoj definiciji obrasca razreda nalaze dva sučelja ovisno o modu rada. Desno je sučelje u modu rada koji korsti radno čekanje dok je lijevo sučelje koje koristi iznimke.
\pagebreak
\begin{multicols}{2}
\lstset{language=C, tabsize=2, frame=single, breaklines=true, showstringspaces=false, basicstyle=\small}
\begin{lstlisting}    
class Driver<Policy::Polling> {
    /*--- Constructor ---*/
    Driver() noexcept;

    /*--- Methods ---*/
    std::uint8_t 
    get() noexcept;
    
    template <std::size_t N>
    std::array<
        std::uint8_t, 
        N
    > 
    get() noexcept;
};
\end{lstlisting}
\columnbreak
\lstset{language=C++, tabsize=1, frame=single, breaklines=true, showstringspaces=false, basicstyle=\small}
\begin{lstlisting}    
class Driver<Policy::Interrupt> {
    /*--- Constructor ---*/
    Driver() noexcept;

    /*--- Methods ---*/
    ConditionStruct 
    set(
    std::invocable<
        std::uint8_t
    >
        auto &userHandler
    ) noexcept;
};
\end{lstlisting}
\end{multicols}
Valja reći kako desno sučelje ima dvije metode, međutim prva navedena je zapravo specijalizacija druge. Prva metoda vraća jedan slučajan broj dok druga metoda vraća njih \textbf{\textit{N}}, korisnik može koristiti što god mu odgovora. S druge strane sučelje koje se bazira na mehanizmu iznimki ima samo jednu metodu koja uzima kao argument funktor\footnote{Funktor je objekt koji se ponaša kao funkcija} koja prima novo izgenerirani broj. Funkcija također vraća tip nad kojim je moguće provjeriti da li je generator izgenerirao dovoljno brojeva te je moguće čekati dok generator ne izgenerira dovoljno brojeva. Kako bi generator znao kada je dovoljno brojeva izgenerirano funktor vraća \textbf{\textit{bool}} vrijednost kojom može odrediti koliko slučajnih brojeva je dovoljno za potrebe korisnika.

\section{Konteskt mjerenja}
Kako je u prijeašnjem poglavlju "\hyperref[section:setup]{Postav}" bilo objašnjeno za testni primjer, tako je u ovom poglavlju konstruiran kontekst mjerenja, odnosno primjer, koji demonstrira korištenje oba od prezentiranih upravljačkih programa. Primjer koji je konstruiran je nadasve jednostavan, upravljački uređaj se koristi kako bi se izgeneriralo 1000 slučajnih brojeva od 0 do 255 te ih se potom sumira i ispisuje njihova suma. Valja napomenuti da ovako konstruiran primjer ne mjeri samo vrijeme generiranje brojeva već i vrijeme trajanja ispisa i vrijeme trajanja sumacije brojeva, međutim jedini dio koji se mjenja unutar ovog primjera jesu implementacije upravljačkih programa sve ostale komponente su konstantne. To je bitno za napomenuti kako se krajnje prezentirani rezultati ne bi smatrali kao absolutno vrijeme generiranja brojeva. Važno je bilo iskonstruirati primjer koji je kompliciraniji od samog generiranja slučajnih brojeva kako bi primjer bio donekle vjerodostojan stvarnoj uporabi. Dodatno valja napomenuti kako su zapravo prevedena tri konteksta za mjerenje iz razloga što se upravljački program u novijoj izvedbi, uz pomoć programskog jezika C++, može koristiti na dva načina te je smatrano kako bi bilo interesantno usporediti vrijeme izvođenja i veličinu krajnjeg programa između ta dva načina korištenja upravljačkog programa. Na kraju valja spomenuti kako sam uređaj za generiranje slučajnih brojeva posjeduje svojstveno vremenske nedeterminističnosti na razini mikrosekunda, odnosno za generiranje jednog broja nije uvijek potrebno identična količina vremena, međutim iako je ovo istina mikrosekunde u slučaju ovog rada nisu presudne u rezultatima mjerenja. Zbog nedeterminističnosti uređaja postojat će određena mala devijacija između vremena međutim to će biti par mikrosekundi razlike koje ne bi trebale previše utjecati na cijelokupno mjerenje.
