\chapter{Izrada upravljačkog programa}
Srž ovog diplomskog rada je izraditi i analizirati upravljački program za određeni periferni uređaj ugradbenog uređaja. Izrađeni upravljački program će biti uspoređen sa implementacijom napisanom u programskom jeziku C od strane proizvođača samog uređaja, ta činjnica bi trebala osigurati najbolju kvalitetu upravljačkog programa. Ekspresivnost izvornog koda je bitan faktor u izradi programskog proizvoda i bitan u ovom diplomskom radu jer je pretpostavka kako su moderne tehnologije ekspresivnije jer imaju bolju sposobnost izricanja inženjerske namjere. Međutim, vrlo je teško objektivno kvatificirati ekspresivnost izvornog koda, tako da će faktor ekspresivnosti biti bez objektivne kvatifikacije, ostavljen na čitatelju da sam procijeni ekspresivnost modernih naspram starih tehnologija. Osim ekspresivnosti izvornog koda, objektivna svojstva od interesa za ovaj diplomski rad su definitivno vrijeme izvođenja i veličina krajnjeg programa. Pretpostavka je kako se ova dva svojstva ne razlikuju između odabranih tehnologija i time dokazuje kako mehanizmi modernih tehnologija ne uvode pesimizaciju ni u jednom od odabranih svojstva. 

\section{Opis platforme}
Platforma koja se koristi za provedbu ovog diplomskog rada je razvojna pločica proizvedena od strane kompanije \textit{Aconno} koja koristi mikrokontroler kompanije \textit{Nordic Semiconductors}. Točan mikrokontroler je \textit{nRF52840} koji je vrlo popular u \textit{IoT}\engl{Internet of things} domeni, prvenstveno zbog podrške bežične komunikacije putem popularnih protokola poput \textit{Wi-Fi}, \textit{Bluetooth}, \textit{BLE}\engl{Bluetooth Low Energy}, itd. Osim mogućnosti bežične komunikacije mikrokontroler je obskrbljen sa mnogo dodatne periferije za procesiranje signala, kriptografiju, serijsku komunikaciju, sinkronizaciju, itd. Sam mikroprocesor je izgrađen po \textit{ARM Cortex-M4} arhitekturi koja dolazi s ugrađenom \textit{FPU}\engl{Floating Point Unit} jedinicom i sklopovljem za otklanjanjem grešaka, te podržava \textit{ARMv7-M} instrukcijski set. \\
Gore navedeni uređaj je odabran zbog svoje široko podržane \textit{ARM} arithekture, vrlo dobre podrške putem \textit{Nordic Semiconductors} foruma i zajednice inženjera, mnogobrojne i močne periferije, lakoće samog programiranja, podrške modernog prevodioca specifično izgrađenim za \textit{ARM} procesore i mogućnosti korištenja sklopovlja za otklanjanje pogrešaka. Sa razvoj programa za sami ugradbeni uređaj koristi se osobno računalo sa operacijskim sustavom \textit{Linux} i ostalim alatima. 

\section{Alati}
Za samo učitavanje krajnjeg programa na ugradbeni uređaj koristi se program komandne linije \textit{nrfjprog} pruženog od strane proizvođača samog mikrokontrolera. Alat je vrlo jednostavan za korištenje i za svoj rad zahtjeva samo spajanje razvojne pločice sa osobnim računalom putem USB kabla. Za razvoj samog izvornog koda koristi se razvojno okruženje \textit{CLion} od tvrtke \textit{JetBrains} koji je vjerojatno jedino dobro razvojno okruženje dostupno na \textit{Linux} operacijskom sustavu za razvoj programa uz pomoć programskog jezika C/C++. Verzioniranje koda je ostvareno uz pomoć alata \textit{git} koji je danas, \textit{de facto}, standard za verzioniranje. Iako \textit{git} dolazi kao komandno linijski alat većinom je korištena ekstenzija razvojnog okruženja \textit{CLion} koja indirektno koristi terminalni alat. Verzioniranje koda je bitna sastavnica projekta, međutim ovaj diplomski rad nema veliku povijest iz čega se da zaključiti kako je alat \textit{git} korišten vrlo oskudno. Veći faktor od do sada spomenutih tehnologija je defnitivno alat za izgradnju programa, \textit{CMake}. \textit{CMake} je ekstenzivno korišten, pokušavajući poštovati moderne smijernice koje dolaze sa modernijim verzijama alata. Valja ispraviti grešku i reči kako nije točno reći da je \textit{CMake} alat za izgradnju alata, već je \textit{CMake} generator za alat za izgradnju alata, što komplicira mentalnu sliku o tome što \textit{CMake} jest, međutim to je istina. \textit{CMake} je korišten kako bi se nadomijestilo stari, još i danas popularan alat, \textit{Make}, koji je poput programskog jezika C, dobar alat, međutim današnji alati nude bolje mehanizme za izgradnju projekata. \textit{CMake} u ovom diplomskom radu interno koristi \textit{Make} kao krajnji alat za izgradnju programa, međutim sam projekt nema nikakve veze sa alatom \textit{Make} te se potencijalno može koristi drugi alati za izgranju programa poput, danas popularnog, \textit{Ninja}-e.\\
Svi dosada nevedeni alati su striktno pomoćni alati u razvoju softvera, alat bez kojega bi svi ovi alati bili suvišni jest prevodilac\engl{Compiler}. Točnije sam prevodilac je alat koji se sastoji od niza alata, zato se ponekad naziva i lanac alata\engl{Toolchain}. U okviru ovog diplomskog rada koristi se \textit{GNU}, C i C++ prevodilac, točnije \textit{ARM GNU} lanac alata verzije 10.3\footnote{Link na stranicu prevodioca: \hyperref[https://developer.arm.com/downloads/-/gnu-rm]{https://developer.arm.com/downloads/-/gnu-rm}}. Ovo je vrlo moderna verzija \textit{GNU} prevodioca koja je potrebna za kontekst ovog diplomskog rada kako bi se pokazale mogućnosti modernog C++. Kako bi se izrađeni upravljački mogao uspoređivati s nečim, dodatno je bilo potrebno uvesti ovisnost o \textit{nRF5 SDK}-u\engl{Software Development Kit} koji sadrži implementacije svog softvera za \textit{Nordic Semiconductors} proizvode, pa tako i za \textit{nRF52840} mikrokontroler. Uz \textit{Nordic Semiconductors} SDK kako bi realizirali povratnu informaciju, sa ugradbenog uređaja nazad na osobno računalo, dodatno je uvedena ovisnost o \textit{Segger RTT} modulu koji omogućava korištenje standardnih funkcija poput \textbf{\textit{printf}} tako da se poruke proslijede osobnom računalu koje koristi alat \textit{JLinkRTTViewer} kao serijski monitor za prihvačanje proslijeđenih poruka. Na taj način korisnik može lakše analizirati ponašanje samog ugradbenog uređaja. Ukoliko je potrebno, koristi se \textit{GDB}\engl{GNU Debugger} zajedno sa \textit{JLinkGDBServer}-om kako bi se zaista duboko analizrala sama egzekucija samog mašinskog koda. Kako bi se analizirali krajnji produkti, objektni kod i krajnji program, koristili su se alati iz paketa alata \textit{binutils} koji dolazi zajedno s \textit{GNU} lancom alata. 