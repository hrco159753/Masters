\chapter{Moderne tehnologije u programskoj potpori za ugradbene računalne sustave}
Stanje danas u razvoju programske potpore za ugradbene sustave je takvo da je još uvijek daleko najpopularnija tehnologija programski jezik C. O ovoj činjnici govori \textit{Dan Saks} u svojoj konferenciskoj prezentaciji\cite{danSaksWritingBetterEmbeddedSoftware}, gdje jasno prikazuje trendove najpopularnijih tehnologija u domeni ugradbenih uređaja gdje C neupitno prednjači nad bilo kojom drugom tehnologijom. Gledajući retrospektivno kroz povijest može se naslutiti zašto je i danas, u 2022. godini, tehnologija izbora mnogih inženjera u domeni ugradbenih uređaja još uvijek programski jezik C, jednostavno povijesno gledano C je bio dugo vremena prisutan i ustalio se u inženjerskoj praksi kao jedini pravi odgovor. Toliko se ustalio u inženjerskom poslu da ga inženjeri diljem svijeta, u različitim domenama, nazivaju \textit{Lingua Franca}. Dodatno, iz tehničke perspektive, programski jezik C je savršeno dizajniran tako da inženjer ima potpunu kontrolu nad svojim programom, te ostavlja vrlo malo prostora za neku drugu tehnologiju, osim možda asemblerskog koda platforme, uz to postoji i C prevodilac za gotovo svaku postojeću platformu. Postoji ogromna količina softvera napisanih u programskom jeziku C, a neki su i napisani kako bi pospiješili produktivnost samo jezika. Gotovo svi popularni operacijski sustavi danas svoje jezgre imaju razvijene uz pomoć C-a, što direktno povlači i činjenicu da postoji potražnja za C programerima. Programski jezik C je fenomen u računarskoj povijesti i vjerojatno će još dugo vremena biti "zajednički nazivnik" svih tehnologija u računarstvu. Možda najveći razlog zašto C neće tako skoro biti zamjenjen nekom drugom tehnologijom je dizajnerska odluka jezika da jezik vjerno prikazuje platformu na kojoj se izvršava. Ukoliko ništa od navedenog ne uvjerava čitatelja kako jezik C tu da ostane, onda je tu činjenica da nove generacije inženjera ne mogu ništa izgubiti s poznavanjem programskog jezika C, što znači da je moguće da će u budućnosti biti korišten u obrazovne svrhe. \\
Sa svim tim rečenim, programski jezik C ima mane i ovaj diplomski rad smatra kako postoje bolje, moderne, alterntive u pojedinim kontekstima. Ne opovrgava kako je programski jezik C validan izbor u mnogim slučajevima, međutim tvrdi da mnogi programeri koji danas koriste C, za svoj trenutni projekt bi trebali koristiti nešto drugo, nešto modernije. Također, prije prelaska u raspravu o modernim tehnologijama, vrijedi napomentuti kako programski jezik C nije jedina starija tehnologija korištena u domeni ugradbenih uređaja, ali je toliko dominantna da se može ugrubo reći da je jedina. Za kontekst ovog rada možemo reći kako je programski jezik C predstavnik starih tehnologija za razvoj programske podrške u domeni ugradbenih uređaja. Moderne tehnologije su zasnovane na prijašnjima kao što je to uvijek istina u inženjerskoj praksi - iteriranjem nad starim idejama dolazimo do novih. Kao i za stare tehnologije, kako bi se olakšala rasprava, za svrhu ovog diplomskog rada odabran je predstavnik modernih tehnologija, očekivano, programski jezik C++, odnosno moderna inačica jezika C++. Ovaj odabir nije slučajan, već je izabran iz razloga što je druga najpopularnija tehnologija navedena u konferenciskoj prezentacija \textit{Dan Saks}-a\cite{danSaksWritingBetterEmbeddedSoftware} upravo programski jezik C++. Iako možda izgleda šaljivo, kroz ovaj diplomski rad pokušati će se prezenirati izazovi i rješenja modernih tehnologija u izradi upravljakih programa u ugradbenim uređajima tako što će se uspoređivati odabrani predstavnici starih i novih tehnologija, programski jezik C i C++. Ovo je napravljeno kako bi se pokušalo doći do konkretnih problema i rješenja koja konkretna moderna tehnologija, ovdje C++, rješava. Osim programskog jezika C++ kao predstavnika modernih tehnologija, vrijedi spomenuti vrlo obečavajuću modernu tehnologiju, programski jezik Rust, koja se predstavlja kao tehnologija koja će zamjeniti programski jezik C. 

\section{Potrebe modernih sustava}

\section{Usporedba tehnologija}